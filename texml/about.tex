\documentclass{amsart}

\title{About}
\author{Jonny Evans}
\email{j.d.evans@ucl.ac.uk}


\begin{document}

When I was an undergraduate, I bought a card file with lots of coloured index cards. I used them to make notes about theorems and proofs. Something appealed about summarising exactly the crucial information into such a constrained space. However, it always seemed that the World-Wide Web was the natural place for such information to live, as hypertext. In its current incarnation, that is what I want Mathematical Field Notes to be: a repository of hypertext index cards about mathematics.

It fulfils a few different roles:
<ul>
<li>There is a lot of mathematical folklore which is either not written down or else widely distributed through the literature. This becomes particularly noticeable when you supervise masters-level or graduate students, and they ask why something is true, or where they can read in depth about something. We usually end up figuring out the answer on the fly and it seems sensible to make a note of it (if only for our future selves).</li>
<li>There's a huge amount of {\em data} in mathematics which is also not easy to find. For example, when my collaborators and I try to do some computations in obstruction theory, we always have to look up the relevant homotopy groups of Lie groups or homogeneous spaces. These are a little scattered in the 1950s/60s literature; it seems useful to digitise some of this information and keep it in XML format.</li>
<li>I think that the optimal way to read a mathematics text book is nonlinearly (you find the information you're looking for and cascade backwards trying to figure out what it means). I hope that the field note format will facilitate this kind of process, so I am trying to write each field note in such a way that it can be understood with a minimum of prerequisites but such that it hyperlinks back to these prerequisites. I will also create some "suggested pathways" through the field notes when there is more content online, which I hope will make it function more like an ordinary textbook (or union of textbooks).</li>
<li>The modularity of the field notes means that they are easier to generate quickly, and for others to contribute to a small part of the whole. It also means it will be easier to locally peer-review material, that is check the correctness/pedagogical or research value of individual field notes or clusters of field notes with little investment of time. I hope that this will lead to a communal, Git-based development of Mathematical Field Notes with contributions (both original and referee reports) from many people.</li>
</ul>

\end{document}
