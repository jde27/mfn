\documentclass{amsart}

\title{Chekanov torus}
\author{Jonny Evans}
\email{j.d.evans@ucl.ac.uk}

%@lagrangian-submanifold.xml
%@symplectic-connection.xml

%#Lagrangians

\begin{document}

\section{Construction}

Consider the Lefschetz fibration $\mathbf{C}^2\stackrel{\pi}{\to}{\mathbf{C}}$, $\pi(x,y)=xy$, and let $\omega$ be the standard symplectic form on $\mathbf{C}^2$. Let $\mathcal{H}$ denote the associated <a link="symplectic-connection.xml" lid="dfnsympconn">symplectic connection</a>.
      
The circle action $\phi_\theta(x,y)=(e^{i\theta}x,e^{-i\theta}y)$ satisfies $\pi(\phi_\theta(x,y))=\pi(x,y)$ and is generated by the <a link="symplectic-vf.htm#hamvf">Hamiltonian vector field</a> $v_H$ associated to $H=\frac{1}{2}\left(|x|^2-|y|^2\right)$. <a link="symplectic-connection.xml" lid="lmaconserved"/> tells us that $H$ is preserved by parallel transport. If $\gamma\colon S^1\to\mathbf{C}$ is an embedded loop disjoint from $\{0\}$ then this means
\[T_{\gamma,h}:=H^{-1}(h)\cap\pi^{-1}(\gamma(S^1))\]
is a torus which is invariant under parallel transport along $\gamma$, for any $h\in\mathbf{R}$. By <a link="symplectic-connection.xml" lid="lmafibred"/>, this is a Lagrangian torus.

\begin{Definition}
Let $\gamma$ be a simple closed curve in $\mathbf{C}$ with winding number zero around $0$. Then $T_{\gamma,h}$ is called a Chekanov torus.
\end{Definition}

\section{Discs}

Let $\gamma$ be a simple closed curve in $\CC$ which does not wind around $0$ and let $\Delta$ be the disc bounded by $\gamma$; by the Riemann mapping theorem, $\Delta$ is biholomorphic to the unit disc. Choose a branch cut starting from $0$ and avoiding $\gamma$ and let $\sqrt{z}$ be a branch of the square root function defined away from this cut. Then for each $\theta$, $u_\theta(z):=\left(\sqrt{z},e^{i\theta}\sqrt{z}\right)$ defines a holomorphic map $u_\theta\colon\left(D,\partial D\right)\to\left(\mathbf{C}^2,T_{\gamma,h}\right)$.
  
\begin{Proposition}[prpchkdisc]
Any holomorphic map $u\colon\left(D,\partial D\right)\to\left(\mathbf{C}^2,T_{\gamma,0}\right)$ with Maslov index 2 has the form $u_{\theta}$, possibly after reparametrisation.
\end{Proposition}
\begin{proof}
Let $u$ be such a holomorphic map. The composition $\pi\circ u\colon (D,\partial D)\to(\mathbf{C},\gamma)$ is a holomorphic disc with boundary on $\gamma$. By the maximum principle, $u(D)\subset\Delta$. Therefore, $u$ defines a relative homology class in $H_2(\pi^{-1}(\Delta),T_{\gamma,0};\mathbf{Z})$. This relative homology group is isomorphic to $H_2(\Delta\times\mathbf{C}^*,T_{\gamma,0};\mathbf{Z})=H_2(\Delta\times S^1,\partial \Delta\times S^1;\mathbf{Z})=\mathbf{Z}$. Therefore there is a unique relative homology class of discs with positive area and Maslov index 2 which can contain holomorphic discs. The discs $u_{\theta}$ live in this class. They are sections of $\pi$, therefore Maslov index 2 holomorphic discs on $T_{\gamma,0}$ project with degree one to $\Delta$. Degree one maps $D\to\Delta$ are biholomorphisms. Therefore, possibly after reparametrising, $u(z)=(x(z),y(z))$ satisfies $x(z)y(z)=z$.

Consider the map $\sigma\colon\mathbf{C}^2\setminus\pi^{-1}(0)\to\mathbf{C}^*$, $\sigma(x,y)=y/x$. If $(x,y)\in T_{\gamma,0}$, then $|x|^2=|y|^2$, so $|\sigma(x,y)|=1$. The composition $\sigma\circ u$ therefore defines a map $(D,\partial D)\to (\mathbf{C}^*,U(1))$, where $U(1)$ is the unit circle. This means that $\sigma\circ u$ is a constant $e^{i\theta}\in S^1$. Therefore if $u(z)=(x(z),y(z))$ then $y(z)/x(z)=e^{i\theta}\in U(1)$.

Since $z=x(z)y(z)=x(z)^2e^{i\theta}$, we get that $(x(z),y(z))=(e^{-i\theta/2}\sqrt{z},e^{i\theta/2}\sqrt{z})$. Reparametrising $z\mapsto e^{i\theta}z$, we end up with
\[u(z)=(x(z),y(z))=(\sqrt{z},e^{i\theta}\sqrt{z})=u_\theta(z)\]
as required.
\end{proof}

\end{document}
