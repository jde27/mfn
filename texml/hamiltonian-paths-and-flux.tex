\documentclass{article}

\title{Hamiltonian paths and flux}
\author{Jonny Evans}

%@flux.xml
%@symplectic-connection.xml

%#Flux

\begin{document}

\section{Hamiltonian paths and flux}

\begin{Lemma}[lmafluxham]
  Let $\psi=\{\psi_t\}_{t=0}^T$ be a path of symplectomorphisms with $\psi_0=\mathrm{id}$. Then $\psi_t$ is Hamiltonian for all $t\in[0,T]$ if and only if $\mathfrak{flux}(\psi)_0^s=0$ for all $s\in[0,T]$.
\end{Lemma}
\begin{proof}
  We have
  \[\frac{d}{ds}\mathfrak{flux}(\psi)_0^s=[\iota_{X_s}\omega]\]
  which vanishes for all $s$ if and only if $\iota_{X_s}\omega=-dH_s$ for all $s$.
\end{proof}

\begin{Proposition}[thmfluxham]
  If $\psi=\{\psi_t\}_{t=0}^1$ is a path of symplectomorphisms with $\mathfrak{flux}(\psi)_0^1=0$ then $\psi$ is homotopic (rel endpoints) to a path of Hamiltonian symplectomorphisms.
\end{Proposition}
\begin{proof}
  By <a lid="lmafluxham"/> it suffices to homotope $\psi$ to kill off all of its flux. Pick a basis $e_1,\ldots,e_n$ for $H^1(X;\RR)$ and, for each basis vector $e_i$, pick a symplectic vector field $Y_i$ such that $[\iota_{Y_i}\omega]=e_i$. Let $\phi_i^t$ denote the Hamiltonian flow of $Y_i$ for time $t$. Let us write
  \[\mathfrak{flux}(\psi)_0^t=\sum f_i(t)e_i\]
  and set $\theta^s_t=\phi_1^{sf_1(t)}\circ\cdots\circ\phi_n^{sf_n(t)}\circ\psi_t$. This defines a family of symplectic isotopies $\theta^s=\{\theta^s_t\}_{t=0}^1$ with $\theta^0=\psi$, $\theta^s_0=\mathrm{id}$ and $\theta^s_1=\psi_1$ for all $s$ (since $f_i(1)=0$ for all $i$ by assumption). Moreover $\theta^s$ has flux equal to
  \[\mathfrak{flux}(\theta^s)_0^t=\mathfrak{flux}(\psi)_0^t-s\sum\mathfrak{flux}(\phi_i)_0^t=\sum_{i=1}^n(f_i(t)-sf_i(t))e_i,\]
  so $\mathfrak{flux}(\theta^1)_0^t=0$ for all $t$ and $\theta^1$ is a Hamiltonian isotopy.
\end{proof}

\end{document}
