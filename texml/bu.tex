\documentclass{amsart}

\title{$BU(n)$}
\author{Jonny Evans}
\email{j.d.evans@ucl.ac.uk}

%@classifying-spaces.xml
%@chern-classes.xml
%@characteristic-classes.xml
%@grassmannians.xml

%#TopologyOfLieGroups
%#Bundles

\begin{document}

\section{Definitions and examples}

The space $BU(n)$ is the direct limit of Grassmannians $\lim_{N\to\infty}\mathrm{Gr}(n,N)$. It is the <a link="classifying-spaces.xml" lid="dfn2">classifying space</a> for complex vector bundles of rank $n$. Equivalently, $BU(n)=EU(n)/U(n)$ where $EU(n)$ is a contractible free $U(n)$-space (such an $EU(n)$ is called the universal principal $U(n)$-bundle).

\begin{Example}[exm1]
  Consider the case $n=1$. There is a free $U(1)$ action on $S^{2N-1}\subset\CC^N$ for each $N$: $(z_1,\ldots,z_N)\mapsto(e^{i\theta}z_1,\ldots,e^{i\theta}z_N)$, $\sum_{i=1}^N|z_i|^2=1$. Define $EU(1):=S^{\infty}:=\lim_{N\to\infty}S^{2N-1}$; this still has a free $U(1)$-action, but it is also contractible.
\end{Example}
\begin{proof}
  To see this, observe that the $k$th homotopy group of $S^{\infty}$ is the limit of $\pi_k(S^{2N-1})=0$, so the inclusion of a point into $S^{\infty}$ is a weak homotopy equivalence, and since $S^{\infty}$ has the homotopy type of a connected, countable CW-complex, <a link="https://en.wikipedia.org/wiki/Whitehead_theorem">Whitehead's theorem</a> tells us that this is actually a homotopy equivalence.

  Alternatively, one can construct a nullhomotopy explicitly. This is done in Hatcher's book.
\end{proof}

Therefore $BU(1)=\lim_{N\to\infty}S^{2N-1}/U(1)=\lim_{N\to\infty}\CC\mathbf{P}^N$ is the direct limit of <a link="cpn.xml">complex projective spaces</a>.

We can use this to compute its homotopy and cohomology groups...TBC

\section{Characteristic classes}

\begin{Definition}
  A characteristic class for rank $k$ complex vector bundles is a cohomology class of $BU(k)$. If $E\to X$ is a rank $k$ complex vector bundle over a compact Hausdorff space $X$ then let $f\colon X\to BU(k)$ be the (unique up to homotopy) classifying map for $E$. For each cohomology class $c\in H^*(BU(k))$, define the corresponding characteristic class of $E$ to be $f^*c\in H^*(X)$.
\end{Definition}

\begin{Example}
  We have $H^*(BU(k);\ZZ)=\ZZ[c_1,c_2,\ldots,c_k]$ for some generators $c_i\in H^{2i}(BU(k);\ZZ)$. The characteristic class corresponding to $c_i$ is called the $i$th Chern class.
\end{Example}

\end{document}
