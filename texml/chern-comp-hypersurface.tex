\document{article}

\title{Chern classes for hypersurfaces in $\CC P^n$}
\author{Jonny Evans}

%#WorkedExample

%@chern-class.xml
%@characteristic-class.xml

\begin{document}

\section{Computation of Chern classes}

Let $X\subset\CC P^n$ be a smooth hypersurface of degree $d$. Let $H$ denote the cohomology class in $H^2(\CC P^n;\ZZ)$ Poincare dual to a hyperplane in $\CC P^n$ and $H|_X$ denote the cohomology class in $H^2(X;\ZZ)$ Poincare dual to a hyperplane section of $X$.

\begin{Lemma}
  The normal bundle $\nu X$ of $X$ in $\CC P^n$ has first Chern class $dH|_X$.
\end{Lemma}
\begin{proof}
  The vanishing locus of a section of $\nu X$ is Poincare dual to $c_1(\nu X)$. If $X'$ is another hypersurface of degree $d$ which is $\mathcal{C}^1$-close to $X$ then it defines a section of $\nu X$. The vanishing locus of this section is the intersection between $X$ and $X'$ which is homologous to $dH|_X$ as $X'$ is homologous to $dH$.
\end{proof}

The total Chern class of a vector bundle $E\to B$ is the formal sum
\[c(E)=1+c_1(E)+c_2(E)+\cdots\in H^*(B;\ZZ)\]
living in mixed degree in the cohomology of $B$. Recall the <a link="chern-class.xml" lid="whitney">Whitney sum formula</a>: if $E\cong F\oplus F'$ as complex vector bundles then
\[c(E)=c(F)c(F').\]
In the hypersurface setting, since $\nu X=(T\CC P^n|_X)/TX$, we have
\[c(T\CC P^n|_X)=c(TX)c(\nu X).\]
We use this, first to find the total Chern class of $T\CC P^n$, then to find the total Chern class of $TX$.

\begin{Corollary}[corcherncp]
  Taking $d=1$, so that $X=\CC P^{n-1}$, we see that
  \[c(T\CC P^n|_X)=c(T\CC P^{n-1})(1+H|_X)\]
  so $c(T\CC P^n)=(1+H)^{n+1}$ by induction. In particular,
  \[c_1(\CC P^n)=(n+1)H,\quad c_2(\CC P^n)=\binom{n+1}{2}H^2.\]
\end{Corollary}

\begin{Corollary}
  If $X$ is a hypersurface of degree $d$,
  \[(1+H)^n|_X=c(TX)(1+dH|_X),\]
  which determines $c(TX)$ by comparing terms order by order. In particular,
  \[c_1(X)=(n+1-d)H|_X\]
  so $X$ is Fano if $d\leq n$, Calabi-Yau if $d=n+1$ and general type if $d>n+1$.
\end{Corollary}

\begin{Corollary}
  When $n=2$ and $X$ is a curve of degree $d$ in $\CC P^2$, $H|_X$ is a collection of $d$ points in $X$ and we deduce the degree-genus formula
  \[2-2g(X)=c_1(X)=(3-d)d\]
  or
  \[g(X)=\frac{2-3d+d^2}{2}=\frac{(d-1)(d-2)}{2}.\]
\end{Corollary}

\end{document}
