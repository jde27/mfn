\title{Classifying spaces for complex vector bundles}

\author{Jonny Evans}
\email{j.d.evans@ucl.ac.uk}

%@bu.xml
%@grassmannian.xml
%@vector-bundle.xml
%@fibre-bundle.xml
%@characteristic-class.xml
%@chern-class.xml

%#TopologyOfLieGroups
%#Bundles

\begin{document}

\section{Existence of classifying maps}

Let $X$ be a compact Hausdorff space and suppose that $E\to X$ is a rank $k$ complex vector bundle over $X$.
\begin{Lemma}[lma1]
There exists an inclusion $\alpha\colon E\to\underline{\CC^N}$ of $E$ as a subbundle of the trivial bundle $\underline{\mathbf{C}^N}$ for some $N$.
\end{Lemma}
\begin{proof}
By compactness, we can find a finite collection of open sets $U_i\subset X$, $i=1,\ldots,n$ such that $E|_{U_i}$ can be trivialised. For each $U_i$, let $\sigma_{i,1},\ldots,\sigma_{i,k}$ be sections of $E|_{U_i}$ such that $\sigma_{i,1}(x),\ldots,\sigma_{i,k}(x)$ form a basis of the fibre $E_x$ at every $x\in U_i$. Then every $v\in E_x$ can be written in a unique way as $v=\sum_{j=1}^kv_j\sigma_{i,j}(x)$. We define a map $a_i\colon E|_{U_i}\to X\times\mathbf{C}^k$ by
\[a_i(v)=(x,v_1,\ldots,v_k)\mbox{ (where }v\in E_x\mbox{).}\]
Let $\rho_i\colon U_i\to\mathbf{R}$ be a partition of unity subordinate to this cover. Then define $\alpha\colon E\to X\times\mathbf{C}^{kn}$ by
\[\alpha(v)=(x,\rho_1(x)a_1(v),\ldots,\rho_n(x)a_n(v))\mbox{ (where }v\in E_x\mbox{)}\]
This makes sense as an expression because $\rho_j$ is supported in $U_j$, which is where $a_j$ is defined. It is an embedding because at every point $x$ one of the $\rho_j$ is nonzero and the corresponding $a_j$ embeds $E_x$ into the corresponding factor of $\mathbf{C}^{kn}$.
\end{proof}

\begin{Definition}[dfn1]
An inclusion $\alpha$, as in the lemma, picks out a copy of $\mathbf{C}^k\subset\mathbf{C}^N$ for each $x\in X$ (the fibre of $E$ at $x$) and therefore we get a map $f\colon X\to\mathrm{Gr}(k,N)$, where $\mathrm{Gr}(k,N)$ is the <a link="grassmannian.xml">Grassmannian</a> of complex $k$-planes in $\CC^N$. We call this a {\em classifying map} for $E$.
\end{Definition}

Let $E(k,N)$ denote the <a link="grassmannian.xml" lid="tautologicalbundle">tautological bundle</a> over $\mathrm{Gr}(k,N)$. By construction,
\[f^*E(k,N)\cong E\]
Since there are natural embeddings
\[\cdots\to\mathrm{Gr}(k,N)\to\mathrm{Gr}(k,N+1)\to\cdots,\]
you can make sense of the limit as $N\to\infty$.
\begin{Definition}[dfn2]
Define the {\em classifying space} and the {\em universal bundle} over it by
\[BU(k):=\lim_{N\to\infty}\mathrm{Gr}(k,N)\qquad E(k)=\lim_{N\to\infty}E(k,N)\]
For some further discussion of the spaces $BU(k)$, in particular about their topology, see <a link="bu.xml">here</a>.
\end{Definition}
<a link="classifying-spaces.xml" lid="lma1"/> then implies that any bundle $E$ is isomorphic to $f^*E(k)$ for some map $f\colon X\to BU(k)$.

\section{Uniqueness of classifying maps}

\begin{Lemma}[lma2]
Two maps $f_0,f_1\colon X\to BU(k)$ are homotopic if and only if $f_0^*E(k)\cong f_1^*E(k)$.
\end{Lemma}
\begin{proof}
Only if: Pull-back of bundles along homotopic maps are always isomorphic by <a link="basic-bundles.xml" lid="lma1"/>.
 
If: Since $X$ is compact, the maps $f_i$ land in $\mathrm{Gr}(k,N_i)$ for some $N_i$. Let $E_i=f_i^*E(k,N_i)$ and let $\alpha_i\colon E_i\to \underline{\CC^{N_i}}$ be the inclusion maps from <a link="classifying-spaces.xml" lid="lma1"/>. The inclusions $(\alpha_0,0)$ and $(0,\alpha_1)$ embed $E_0$ and $E_1$ into $\underline{\CC^{N_0+N_1}}$. There is a homotopy of inclusion maps
\[\alpha_t=((1-t)\alpha_0,t\alpha_1)\colon E\to\underline{\CC^{N_0}}\oplus\underline{\CC^{N_1}}\]
which induces a homotopy of classifying maps $f_t\colon X\to\mathrm{Gr}(k,N_0+N_1)$ between $f_0$ and $f_1$, proving the lemma.
  
One small point: we tacitly used the fact that the inclusion $(0,\alpha_1)$ gives a classifying map which is homotopic to $f_1$. Since $f_1$ is the classifying map obtained from the inclusion $(\alpha_1,0)$, it suffices to show that $(0,\alpha_1)$ and $(\alpha_1,0)$ are homotopic as inclusions $E_1\to\underline{\CC^{N_0+N_1}}$. This is true because the linear maps which permute coordinates of $\CC^N$ are isotopic to the identity in $GL(N,\CC)$.
\end{proof}
Overall this tells us:
\begin{Theorem}
The map $[f]\mapsto f^*E(k)$ defines a bijection between the set of homotopy classes of maps $X\to BU(k)$ and the set of isomorphism classes of complex rank $k$ vector bundles over $X$.
\end{Theorem}
\end{document}
