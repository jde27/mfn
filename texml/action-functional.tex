\documentclass{amsart}

\title{The action functional}
\author{Jonny Evans}
\email{j.d.evans@ucl.ac.uk}

%#FloerTheory
%#Lagrangians

\begin{document}

\section{Exact Lagrangians}

Let $(X,\omega=d\lambda)$ be an exact symplectic manifold.

\begin{Definition}
A Lagrangian embedding $i\colon L\to X$ is {\em exact} if $i^*\lambda=df$ for some function $f\colon L\to\RR$.
\end{Definition}

\begin{Remark}[stokes]
The great advantage of exact manifolds and exact Lagrangians is that, if $J$ is a compatible almost complex structure, there can be no nonconstant holomorphic maps from a compact Riemann surface $\Sigma$ to $X$ sending $\partial\Sigma$ to $L$.
\end{Remark}
\begin{proof}
The $\omega$-area $\int_{\Sigma} u^*\omega$ of such a holomorphic curve $u$ is nonnegative and is zero if and only if $u$ is constant. By Stokes's theorem, exactness implies that the integral is zero.
\end{proof}

\section{Action on paths}

\begin{Definition}
Let $\iota_0\colon L_0\to X$ and $\iota_1\colon L_1\to X$ be exact Lagrangian submanifolds, with $\iota_i^*\lambda=df_i$, and let $P(L_0,L_1)$ denote the space of smooth paths $\gamma\colon[0,1]\to X$ with $\gamma(i)\in L_i$, $i=0,1$. The action functional $A\colon P(L_0,L_1)\to\RR$ is the function
\begin{align*}
A(\gamma)&=\int_0^1\gamma^*\lambda+f_1(\gamma(1))-f_0(\gamma(0))\\
&=\int_0^1\lambda(\dot{\gamma}(t))dt+f_1(\gamma(1))-f_0(\gamma(0))
\end{align*}
\end{Definition}

A variation of a path $\gamma\in P(L_0,L_1)$ is a vector field $v(t)$ along $\gamma$ (i.e. a section of $\gamma^*TX$) such that $v(i)\in TL_i$, $i=0,1$. We think of the space of variations as the tangent space to $P(L_0,L_1)$ at $\gamma$.

\begin{Lemma}
Let $v$ be a variation of $\gamma$. The directional (i.e. Gateaux) derivative of $A$ in the $v$-direction is
\[d_{\gamma}A(v)=\int_0^1\omega(\dot{\gamma}(t),v(t))dt.\]
In particular, this vanishes for all variations $v$ if and only if $\gamma$ is constant ($\dot{\gamma}(t)=0$ for all $t$). This means that the critical points of $A$ are the constant paths at points in $L_0\cap L_1$.
\end{Lemma}
\begin{proof}
If $v$ is a variation then $F(s,t):=\gamma(t)+sv(t)$ is a family of curves parametrised by $s$ (here we're using a coordinate chart around $\gamma$ so we can add points and vectors). The family of curves traces out a surface $S=F([0,\epsilon]\times[0,1])$ and, by Stokes's theorem, the $\omega$-area of $S$ is
\[\int_{[0,\epsilon]\times[0,1]}F^*\omega=\int_{\partial([0,\epsilon]\times[0,1])}F^*\lambda=A(\gamma+\epsilon v)-A(\gamma).\]
We therefore have
\begin{align*}
  d_{\gamma}A(v)&=\lim_{\epsilon\to 0}\frac{1}{\epsilon}\int_SF^*\omega\\
  &=\lim_{\epsilon\to 0}\int_0^1\frac{1}{\epsilon}\int_0^{\epsilon}\omega(\dot{\gamma}+s\dot{v},v)ds dt\\
  &=\int_0^1\omega(\dot{\gamma},v)dt
\end{align*}
because the term $\frac{1}{\epsilon}\int_0^1\int_0^{\epsilon}sds\omega(\dot{v},v)dt$ goes to zero.
\end{proof}

\end{document}
