\documentclass{amsart}

\title{Superpotentials}
\author{Jonny Evans}
\email{j.d.evans@ucl.ac.uk}

%@lagrangian-submanifold.xml

%#FloerTheory
%#Lagrangians

\begin{document}

\section{The superpotential}

Let $X$ be a monotone symplectic $2n$-manifold and let $L\subset X$ be a monotone Lagrangian $n$-torus. Pick a basis $e_1,\ldots,e_n$ for $\ZZ^n=H_1(L;\ZZ)$ and identify $H^1(L;\CC^*)=\mathrm{Hom}(H_1(L;\ZZ),\CC^*)$ with $(\CC^*)^n$ by sending $\underline{z}=(z_1,\ldots,z_n)$ to the homomorphism $e_i\mapsto z_i$. If $\beta$ is a disc with boundary $\partial\beta$ on $L$ then write $\partial\beta=\sum_{i=1}^nm_ie_i$ and $\underline{z}^{\partial\beta}=z_1^{m_1}\cdots z_n^{m_n}$.

\begin{Definition}
  Let $J$ be a (sufficiently generic) $\omega$-compatible almost complex structure. The superpotential associated to $L$ is a holomorphic function (Laurent polynomial)
  \[W\colon H^1(L;\CC^*)\to\CC\]
  defined as follows. For each relative homology class $\beta\in\pi_2(X,L)$ with $\mu(\beta)=2$, let $n(\beta)$ denote the degree of the evaluation map $\mathcal{M}_{0,1}(\beta,J)\to L$ (roughly speaking, the number of $J$-holomorphic discs in the class $\beta$ passing through a generic point in $L$, counted with multiplicity). Define
  \[W(\underline{z})=\sum_{\mu(\beta)=2}n(\beta)\underline{z}^{\partial\beta}.\]
\end{Definition}

Genericity of $J$ is required to make this moduli space $\mathcal{M}_{0,1}(\beta,J)$ transverse and to get transversality for the evaluation map. Note that since $L$ is monotone and orientable, the Maslov 2 discs have minimal area amongst all discs, so this moduli space $\mathcal{M}_{0,1}(\beta,J)$ is compact (no smaller discs can bubble off) and the degree of the evaluation map is well-defined because the dimension of $\mathcal{M}_{0,1}(\beta,J)$ is $\mu+n+1-3=n$ which equals the dimension of $L$.

\begin{Example}
  Consider the equator $L$ on $S^2$. There are two Maslov 2 disc classes (the two hemispheres) each containing one holomorphic disc. The boundaries wrap once (respectively minus once) around $L$ so the associated holonomy contributions to $W$ are $z$ and $z^{-1}$ so the superpotential is $W(z)=z+\frac{1}{z}$.
\end{Example}

\section{Critical points and Floer cohomology}

If $\underline{z}\in H^1(L;\CC^*)$ then we can define a Floer cohomology group $HF((L,\underline{z}),(L,\underline{z}))$ by taking the pearl complex but weighting any pearly trajectory whose boundary is homologous to a loop $\gamma$ by $\underline{z}^{\gamma}$.

\begin{Theorem}
  $HF((L,\underline{z}),(L,\underline{z}))$ is isomorphic to the ordinary homology of $L$ if and only if $\underline{z}$ is a critical point of $W$.
\end{Theorem}
\begin{proof}
  Since $L$ is a torus, its ordinary cohomology is generated by classes in degree 1 (it is an exterior algebra on $H^1(L)$). Since the Biran-Oh spectral sequence is multiplicative, if the Floer differential vanishes on classes of degree one, it vanishes identically. The differential on classes of degree one can only have terms coming from Maslov 2 discs because the Morse differential is zero (if we start with a perfect Morse function on a torus) and any higher Maslov discs take a critical point of index 1 to a critical point of negative index. Indeed, if $q_i$ is the critical point corresponding to the basis element $e^*_i\in H^1(L;\ZZ)$, the differential of $q_i$ equals
  \[\partial q_i=\sum_{\mu(\beta)=2}n(\beta)m_i\underline{z}^{\partial\beta}\mathrm{min}\]
  where $\mathrm{min}$ denotes the minimum. To see this, note that the pearly trajectories contributing to this part of the differential are upward flowlines from $q_i$ which hit a Maslov 2 disc passing through the minimum. Each disc through the minimum intersects the upward manifold of $q_i$ a total of $m_i$ times with multiplicity (as the boundary of the disc $\partial\beta$ is $\sum m_ie_i$ in homology). Therefore $\partial q_i=0$ if and only if $z_i\frac{\partial W}{\partial z_i}=0$.

  If the Floer differential is nonzero then $\partial q_i\neq 0$ for some $i$, hence $\partial q_i$ is a multiple of the minimum. But the minimum represents the identity element in Floer cohomology, hence it is exact if and only if the Floer cohomology is nonzero.
  
  We deduce that Floer cohomology is either zero or isomorphic to ordinary cohomology, and this latter case happens if and only if $\underline{z}$ is a critical point of $W$.
\end{proof}

\end{document}
