\documentclass{amsart}

\title{The bundle $\mathcal{O}(-1)$}
\author{Jonny Evans}
\email{j.d.evans@ucl.ac.uk}

%@classifying-spaces.xml
%@vector-bundle.xml
%@fibre-bundle.xml
%@gr-complex.xml
%@gr-real.xml
%@cpn.xml

%#TopologyOfLieGroups
%#Bundles

\begin{document}

\section{Total space}

Consider homogeneous coordinates $[x:y]$ on $\CC P^1$. The coordinates $x$ and $y$ themselves are not well-defined functions on $\CC P^1$: each point $[x_0:y_0]$ can be described by any pair $x,y$ such that the ratio of $x$ to $y$ equals the ratio of $x_0$ to $y_0$. However, $x$ and $y$ do make sense as functions on a different space:

\begin{Definition}[dfntautbun]
Define $\mathcal{O}(-1)$ to be the following variety
\[\mathcal{O}(-1):=\{(x,y,[a:b])\in\CC^2\times\CC P^1\ :\ ay=bx\}\subset\CC^2\times\CC P^1.\]
This variety contains a Zariski open set $U=\{(x,y,[x:y])\ :\ (x,y)\neq (0,0)\}$. The complement of this Zariski open set is the set of points $\{(0,0)\}\times\CC P^1$. Indeed one could also define it as the Zariski closure of $U$.
\end{Definition}
This variety admits projections $\pi\colon\mathcal{O}(-1)\to\CC P^1$, $\pi(x,y,[a:b])=[a:b]$, and $F=(F_1,F_2)\colon\mathcal{O}(-1)\to\CC^2$, $F_1(x,y,[a:b])=x$, $F_2(x,y,[a:b])=y$. In particular, $x$ and $y$ make sense as functions $F_1$ and $F_2$ on the total space of $\mathcal{O}(-1)$.

\section{Sections}

The projection $\pi$ exhibits $\mathcal{O}(-1)$ as the total space of a complex line bundle over $\CC P^1$: the fibre over $[a:b]$ is the set of all points $(\lambda a,\lambda b)$, $\lambda\in\CC$. Recall that $\CC P^1$ parametrises complex lines in $\CC^2$ passing through the origin, so that the point $[a:b]$ parametrises the line $\{(\lambda a,\lambda b)\ :\ \lambda\in\CC\}$. Therefore the fibre $\pi^{-1}([a:b])$ is precisely the complex line in $\CC^2$ which is labelled $[a:b]$ in homogeneous coordinates! For this reason, we call $\mathcal{O}(1)$ the {\em tautological line bundle} over $\CC P^1$.

\begin{Lemma}[lmatautbun]
$\mathcal{O}(-1)$ has no global holomorphic sections except the zero section. (The zero section is the section $\sigma\colon\CC P^1\to\mathcal{O}(-1)$ given by $\sigma([a:b])=(0,0,[a:b])$.)
\end{Lemma}
\begin{proof}
Suppose there were another global holomorphic section, i.e. a map $\CC P^1\to \mathcal{O}(-1)$ such that $\pi\circ\sigma([a:b])=[a:b]$. Then
\[aF_2(\sigma([a:b]))=bF_1(\sigma([a:b])).\]
If we are assuming this is not the zero section then there exists a point $[a:b]$ (and hence an open set around that point) where either $F_1(\sigma([a:b]))$ or $F_2(\sigma([a:b]))$ is nonzero (WLOG say $F_2)$. This means that $\frac{F_1\circ\sigma}{F_2\circ\sigma}$ is a well-defined holomorphic function on that open set. Since $aF_2=bF_1$, $F_1/F_2=a/b$ where is is defined, and this is nonconstant. In particular, either $F_1\circ\sigma$ or $F_2\circ\sigma$ is nonconstant. But $F_1\circ\sigma$ and $F_2\circ\sigma$ are globally-defined holomorphic functions on $\CC P^1$, hence constant by Liouville's theorem.
\end{proof}

\end{document}
